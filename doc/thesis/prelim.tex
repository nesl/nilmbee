
\title {Low-cost Appliance State Sensing \\ for Energy Disaggregation}
\author {Tianji Wu}
\department {Electrical Engineering}
\degreeyear {2012}

\chair {Mani B.\ Srivastava}
\member {William Kaiser}
\member {Mario Gerla}

\acknowledgments{
In my two years of study and research, I am always feeling grateful to all members of NESL. NESL is a place where people are passionate, freely discuss about the latest technologies, creative ideas, etc. Instead of keeping as words, people do realize those ideas, no matter big or small, passionately. If I had not come to join NESL, I would never have learned so much cutting edge science and technology knowledge and trends and views. If there was only one thing I would miss when I return to my home country, I would miss the geeky environment of NESL. 

Among all, I would especially thank Professor Mani Srivastava, my advisor. Each time when I talked with him, either about academic problem or random ideas, I could feel the inspiration from his broad and deep knowledge and incisive and sharp view, both at architectural level and implementation level. Besides a great academic advisor, he is also a great mentor, helping us to set high expectations in life. 

I would also thank my collaborator Kanthi. Although we have worked together only for couple of weeks, her inspiring ideas and great work have been important to this work. Also thank all NESL members whom I bothered with deploying the sensors, and Fe, without her hard work the whole lab can not run smoothly. 

Finally, I would express my appreciation to my parents, who, despite thousands of miles away, never stop thinking of me; to Boss Wangs, who took care of everything back in China so that I could focus on this work; and to my beloved Catherine, who brings warmness and tenderness to the strict and busy world. 
}

\abstract {
Fine-grained per appliance electrical energy consumption data is crucial to electrical energy conservation. However, energy meters are installed at few central points in buildings, providing only aggregated energy consumption data. Therefore, people are seeking ways to get disaggregated energy information. We design a sensing system that can reliably keep track of the on-off power state of appliances, which is a key information and also most challenging problem in energy disaggregation approaches. 

In our system, the sensors are deployed at each appliance. Even though, the whole system is totally affordable for large scale deployment, thanks to the inherently low-cost sensors we design. The evaluation shows that, despite the simplicity of hardware, the system can keep track of the power state of tens of appliances at 99.5\% precision and recall with a single base station. We also propose an energy disaggregation approach that make use of the power state data combined with central power meter readings. Experiments show acceptable accuracy. 
}
