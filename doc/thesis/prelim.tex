
\title {Low-cost Appliance State Sensing \\ for Energy Disaggregation}
\author {Tianji Wu}
\department {Electrical Engineering}
\degreeyear {2012}

\chair {Mani B.\ Srivastava}
\member {William Kaiser}
\member {Mario Gerla}

\acknowledgments{
In my two years of study and research, I am always feeling grateful to all members of NESL. NESL is a place where people are passionate, discuss about latest technologies, creative ideas. And instead of stoping after discussion, people do realize those ideas, no matter big or small. If I had not come to join NESL, I would never have learned so much cutting edge science and technology knowledge and trends. If there was only one thing I would miss when I return to my home country, I would miss the geeky environment of NESL. 

I would especially thank my advisor, Professor Mani Srivastava. Each time when I talked with him, either about academic problem or new ideas, I could feel the inspiration I got from his broad and deep knowledge and incisive and sharp view, both at lower implementation level and at higher architectural level. Besides a great academic advisor, he is also a great mentor in life, helping us to set high expectations in life. 

I would also thank collaborator Kanthi. Although we have worked together only for couple of weeks, her inspiring ideas and great work have been important to this work. Also thank all NESL members whom I bothered with deploying the sensors, and Fe, without her hard work the whole lab can not run smoothly. 

Finally, I would express my appreciation to my parents, who, despite thousands of miles away, never stop thinking of me; to Boss Wangs, who took care of everything back in China so that I could focus on my work; and to my beloved Catherine. 
}

\abstract {
Fine-grained appliance-level electrical energy consumption data is crucial for electrical energy conservation. Because energy meters are usually installed at a central point, people are seeking ways to get disaggregated energy information. In this work, we design a sensing system that can reliably keep track of the on-off power state of appliances, which is a key information and also most challenging problem in energy disaggregation approaches. 

Although the sensors are deployed at each appliance, the whole system is totally affordable, thanks to the inherently low-cost sensors we designed. The evaluation shows that, despite the simplicity of hardware, the system can keep track of the power state of tens of appliances at 99.5\% precision and recall. We also propose an energy disaggregation approach that combines the power state data with a central meter readings. Experiments show acceptable accuracy. 
}
