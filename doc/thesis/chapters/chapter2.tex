\chapter{Related work}\label{chap2}

Direct sensing, i.e. metering at every appliances, is an accurate approach to energy disaggregation, yet a costly one. Commercial products range from low-end LCD display meters (e.g. Kill-A-Watt) that cost around 20 dollars to high-end networked meters (e.g. Watts up? PRO) that cost around 200 dollars. There are also great research platforms such as ACme\cite{Jiang2009} that can meter and wirelessly transmit energy data. However, their 60-dollar cost still makes it unaffordable for large deployment other than research purposes. 

In recent year, a lot of research work has focused on the other extreme of solutions: NILM, i.e. having no sub-meters, and putting all the intelligence in a central meter. 

NILM, or most of energy disaggregation methods without direct metering, have generally three phases: 1) power state change detection, i.e. detecting that one or more appliances' power consumption has changed. For most appliances, a change in its power consumption means a change in its power state. For example, a microwave oven goes from off to on, or a electric fan goes from high speed to low speed. There are also some appliances that do not have discrete power states, such as a battery charger, whose consumption varies continuously during the charging cycle, or a computer, whose consumption varies with different workload. 2) Appliance identification, i.e. identify which appliance has a state change. This is done with the help of some prior knowledge about each appliances interested. In some NILM algorithms, this phase is done together with the first phase. 3) Estimation of energy numbers for each appliances, i.e. keeping track of disaggregated power of each appliances, with the knowledge of which appliance has caused which power state changes.

The methods of NILM can be generally classified into two categories. The first deals with the statistical values of the voltage and current traces. Such statistical values include RMS voltage, RMS current, apparent power, real power (P), reactive power (Q), power factor (PF), etc. These values are usually obtained at a low sampling frequency, typically less than 1Hz. 
The original NILM method\cite{Hart1992} falls in this category, step changes in aggregated real and reactive power (P and Q) are detected and clustered, so that similar changes are identified to be caused by the same appliance. 