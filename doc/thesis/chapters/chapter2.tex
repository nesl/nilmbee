\chapter{Related work}\label{chap2}

Direct sensing, i.e. metering at every appliance, is the most accurate approach we can expect towards energy disaggregation. Yet it is too costly. Commercial products range from low-end LCD display meters (e.g. Kill-A-Watt) that cost around 20 dollars to high-end networked meters (e.g. Watts up? PRO) that cost around 200 dollars. There are also great research platforms such as ACme \cite{Jiang2009} which has metering and control capability and wireless connectivity. However, the 60-dollar cost still makes it unaffordable for large-scale deployment other than research purposes. 

Another kind of approaches attract much attentions in recent years, namely non-intrusive appliance load monitoring (NILM), which is the opposite extreme to direct sensing, i.e. having no sub-meters, and putting all the intelligence in a central meter. 

NILM, or most of energy disaggregation methods without direct metering, have generally three phases: 1) power state change detection, i.e. detecting that one or more appliances' power consumption has changed. Later, we would use \textit{event} and \textit{power state change} interchangeably. Most appliances have few discrete power states. For example, a microwave oven goes from off to on, or an electric fan goes from high speed to low speed. There are also some appliances that do not have discrete power states, such as a battery charger whose consumption varies continuously during the charging cycle, or a computer whose consumption varies with different workload. 2) Appliance identification,  identifying the appliance that has made the events. This needs some prior knowledge about the signatures of all appliances interested. In some NILM algorithms, this phase is done together with the first phase. 3) Estimation of energy consumption for each appliance, i.e. keeping track of disaggregated power of each appliance. This is based on the previous two phases.

The methods of NILM can be generally classified into two categories. Some methods analysis  steady states, such as apparent power, real power (P), reactive power (Q), power factor (PF), etc. These values are usually obtained at a low sampling frequency, typically less than 1Hz. However, some steady states also need high frequency sampling, for example higher harmonic of current trace or power trace. 

The original NILM method \cite{Hart1992} falls in this category. Step changes in aggregated real and reactive power (P and Q) are detected and clustered, so that similar changes are identified to be associated with the same appliance. This method faces the limitation that some appliances have very similar signature on the P-Q plane. Moreover, for non-linear appliances, the reactive power is not linear, meaning that the change in reactive power measured at the central meter is not the change of reactive power of the appliance. Other algorithms such as genetic algorithm and dynamic programming has also been used \cite{Baranski2004,Baranski2004a}. They use P and Q or just P as their signatures. 
Some work also exploits higher harmonics in the steady state power measurements \cite{Laughman2003}. This adds some new signatures to differentiate appliances. 
In order to reduce the number of appliances that need to be distinguished, researchers also tried pushing the meters down to circuit branch level \cite{Marchiori2011}. This helps detecting and identifying events because some of the large appliances are placed on separate circuits. 

On the other hand, some NILM methods identify appliances by their transient signatures during power states change \cite{Leeb1995,Cox2006}. For example, appliances with motors (e.g. refrigerators) usually produce a large spike when turned on, due to the inductive coils. The drawback is that transient signatures are hard to detect when multiple events happen together \cite{Leeb1995}. Recent research uses the Hidden Markov Model to model the power consumption of a power cycle of appliances \cite{Kolter2012}. Therefore, the model captures both transient signatures and the expected length of a power cycle. 

Although current NILM methods work well in certain circumstances, they still have some limitations. The most challenging problem with NILM is that when the number of appliances increases, 1) it is more likely to have appliances with similar signatures; 2) it is more likely to have overlapping events. In both cases, detection and identification of events become significantly hard. Due to these limitations, current NILM systems can hardly work with more than 6-7 appliances. 
To push up the number of appliances, researchers developed a type of hybrid methods. These methods try to augment the central inferencing algorithm with some additional data collected from distributed sensors. These sensor data can hopefully dramatically improve the accuracy or scalability of the system, without raising the cost too much. 

ViridiScope \cite{Kim2009a} uses several kinds of sensors to tell the on-off states of appliances (assuming all appliances have binary states). These sensors include light sensors, sound sensors and magnetometers. With the knowledge of the states of all appliances and the aggregated power readings over time, we can calibrate how much power each appliance use by linear programming. In another related work \cite{Jung2010}, the authors are able to automatically identify where to install sub-meters in order to achieve satisfiable disaggregation accuracy. It assumes availability of reliable on-off state information of all appliances, which is actually the purpose of our work.

In this paper, we presents a low cost sensor network that can deliver reliable on-off state information of appliances. By reducing the hardware capabilities, our sensors are inherently low-cost and suitable for large scale deployment. Yet the reliability and accuracy is not compromised because of the constraints of low-cost hardwares. 

