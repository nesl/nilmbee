\chapter{Introduction}\label{chap1}

Electricity has become one of the most important form of energy in the modern age. In 2010, 41\% of the total energy consumption in the United States is consumed to generate electricity. Among the major sectors, residential and commercial use accounts for more than 70\% of the total electricity end use, and it has been continuously growing for the past 40 years, even when consumptions of other major energy sources, such as natural gas and coal, have become stable over time. \cite{U.S.EnergyInformationAdministration2011}

Traditionally, electricity utility companies report only the aggregated energy use to each household or commercial entity, and electricity power meters are usually installed at few central points. However, in order to conserve energy, accurate and fine-grained energy consumption information is crucial. Hopefully, people can get energy use reports telling them how much energy each appliance consumes, and then they can be able to manage the usage of the appliances accordingly in order to conserve energy. Studies have shown that simply providing instantaneous and disaggregated energy consumption feedback to people can help them conserving energy \cite{Darby2006,Parker2006,Fischer2008}.

%Energy disaggregation usually have three phases: 1) detecting appliance state changes, 2) identifying appliances, 3) infer the energy use of appliances. 

There are several ways to disaggregate energy use into finer granularity. One approach is to install power meters for each and every appliance that we are interested. While it can provide the most accurate results we can expect, this method is too expensive in term of the cost of metering devices and the difficulty of deployment. There are several commercial plug load power meters, such as \textit{Watts up?PRO} and \textit{Kill-A-Watt}. Low-end products with LCD display are usually tens of dollars each, while high-end products with networking capabilities are hundreds. In a typical household where there are tens of appliances, the total cost could be easily reach thousands of dollars. Moreover, these meters usually have a big physical dimension compared to AC outlets and plugs, making it inconvenient to deploy house- or building-wide. 

Due to these limitations of the direct metering method, a kind of new approach called nonintrusive appliance load monitoring (NILM or NIALM) has been developed, and it has attracted great attentions in recent years. The main idea is to use signatures observed solely at the central power meter to infer the state and energy usage of each appliance. We leave detailed discussion about NILM to Chapter \ref{chap2}. This kind of method is promising because it requires minimal deployment efforts. Only a smart central meter is needed and all inferences are based on the signals observed at the central meter. It works well especially with high-consumption appliances such as washing machines, dish washers, microwaves. However, its accuracy degrades significantly when the number of appliances increases. In fact, most of these methods can not accommodate more than 6-7 appliances. They also have trouble differentiating appliances with similar signatures, or appliances that are of the same model. 

A hybrid solution can help remedy the problem. In this paper, we will show that networked sensors deployed at each appliance can assist an intelligent central meter to infer the states as well as energy use of the appliances. We designed a very-low-cost sensor node that can sense and transmit the binary on-off state of an appliance attached onto it. We will show that with such under-designed sensors, the cost can be lowered into an acceptable range for large-scale deployment, and yet they can still provide much crucial information for energy disaggregation. In our experiments, we deploy 20 such sensors, and are able to identify and track the state of all appliances attached to them. With the power readings from the central meter, we are able to infer energy consumption of each individual appliance. The process is fully autonomous, without the need for explicit training. 

The paper will be structured as following. In Chapter \ref{chap2}, we will discuss related work, including various energy disaggregation approaches. The hardware of the sensor we developed will be discussed in Chapter \ref{chap3}. Because we use an unidirectional radio  in our system, a specially designed network protocol is presented in Chapter \ref{chap4}, as well as the evaluations. In Chapter \ref{chap5}, we will discuss our experiments done in a real lab setting. An energy disaggregation algorithm that incorporated with our sensing system is presented as well. Chapter \ref{chap6} concludes the work. 


%Why we need to disaggregate electricity usage among appliances?
%Current approaches: sub-meters, NILM, hybrid
