\chapter{Conclusion}\label{chap6}

In this paper, we propose a new sensing system that can detect and report power state changes of individual appliances. In our system, the sensors are only responsible for on-off state change detection, which makes their hardware requirements minimal. The sensors have a extremely simple microcontroller onboard, along with an uncalibrated sensing frontend, and a transmit-only radio transmitter. We show that the simplicity in requirements allows the sensors to be made at an overall price as low as 4.3 dollars each. This makes it affordable for large scale deployment, where sensors are deployed at each appliance, or even embedded in power outlets or plugs. 

Despite low-cost underpowered, the sensor nodes are sufficient to detect on-off state of appliances accurately. Combined with the robust radio scheme we developed, the system can keep track of the power state of tens of appliances with a single base station. Our evaluation both in simulation and in real settings shows that the accuracy (precision and recall) of power state tracking is more than 99.5\%. The system solves two of the main issues in energy disaggregation: event detection and appliance identification. 

Combining the power state data from our sensing system with a central power meter, we are able to associate power state changes with real Watts numbers. Then, we can keep track of the power consumption of individual appliances, and estimate their disaggregated energy consumption. The accuracy is more than 70\% for most appliances in our challenging 20-appliance disaggregation setup, where most appliances have low and similar power consumption. 

Since knowing the state of appliances is the most challenging problem of a number of energy disaggregation methods, our sensing system can surely assist these methods to achieve higher accuracy or scalability. 

