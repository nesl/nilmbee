\chapter{Conclusion}\label{chap6}

In this work, we proposed a new sensing system that can detect and report power state changes of individual appliances. In our system, the sensors are only responsible for on-off state change detection, which makes their hardware requirement minimum. The sensors have a minimum microcontroller onboard, along with an uncalibrated sensing frontend, and a transmit-only radio transmitter. We show that the inherent simplicity in the sensors allows them to be made at an overall price as low as 4 dollars. This makes it affordable to have the sensors deployed at each appliances, or even embed in power outlets or plugs. We show the possibility of a deeply deployed yet affordable solution. 

Despite the low-cost underpowered sensor nodes, they are sufficient to detect on-off state of an appliance accurately. Combined with the robust radio scheme we developed, the system can keep track of the power state of tens of appliances with a single base station. Our evaluation both in simulation and in real world shows that the accuracy (precision and recall) of power state tracking is more than 99.5\%. The system solves two of the main issue in energy disaggregation: event detection and appliance identification. 

Combining the power state data from our sensing system with a central power meter, we are able to associate power state changes with real watts numbers. Then, we can keep track of the power consumption of individual appliances, and estimate their disaggregated energy consumption. 

Since knowing the state of appliances is the most challenging problem of a number of energy disaggregation methods, our sensing system can surely assist these methods to achieve higher accuracy or scalability. 

